\documentclass[b5paper,papersize,fleqn]{jsarticle}
\usepackage{/Users/yamasakishun/Desktop/ymskarticle}

\begin{document}
\title{}
\author{05-201564 Shun Yamasaki}
\date{\today}
\maketitle

\section{Abstract}
線形連立一次方程式を解くことは周知の通り様々な場面で非常に重要な問題である。線形連立一次方程式は
\begin{eqnarray}
  A\vec{x} = \vec{b}
\end{eqnarray}
と表される。今回は,$vec{x}$自体を知る必要はないが,ある行列$M$を用いて$\vec{x}\dager M \vec{x}$と表されるような$\vec{x}$に関係した量を求めたい場合を考える。$A$が行あたり最大$s$個のnon-zero成分を持つ$N\times N$の行列であり,条件数(condition number)\footnote{xx}を$\kappa $とすると,知られている最速の古典アルゴリズムでは$\vec{x},\vec{x}\dager M \vec{x}$を計算するのに$N\sqrt{\kappa }$の時間スケールで計算できる。一方,今回紹介する量子アルゴリズムでは$\vec{x}\dager M \vec{x}$
を$\log(N),\kappa $の多項式時間スケールで計算することができる。これは古典コンピュータに比べて指数関数的なスピードアップである。

\section{Introduction}
\subsection{Quantum Computer}

量子コンピュータは量子力学を利用して古典コンピュータに実現できない計算を含む計算を実現する装置である。Shorのアルゴリズムのように,ある特定の問題に対しては量子コンピュータは古典コンピュータよりはるかに早い時間で計算できる。この論文では線形連立一次方程式の解の特徴について考えていく。

\subsection{Linear Equation}

線形方程式は科学技術・工学においてほぼ全ての分野で応用され,非常に重要であると言える。古典コンピュータでは$N$次の線形方程式を解くのに,一般的に少なくとも$N$オーダーの時間がかかる。

\subsection{Overview}

この論文では,特定の場合ではAbstractと同じ方程式
\begin{eqnarray}
  A\vec{x} = \vec{b}
\end{eqnarray}
において$A$が行あたり最大$s$個のnon-zero成分を持つ$N\times N$の行列であり,条件数(condition number)を$\kappa $とすると,解$\vec{x}$に関連した値を任意の精度で計算するのに$\log(N),\kappa $の多項式時間スケールかかるということを示している。これは古典コンピュータに比べて指数関数的なスピードアップである。また,典型的なケースでは正確性はあまり求められないことが多い。しかし,condition numberは計算量を著しく増大させ得る。これはこの論文におけるアルゴリズムにおける強い制約となっている。

上の線形方程式と同じ条件で以下,考えていくこととする。まず,$\vec{b}$は抽象的な状態$\ket{b}$の行列表現であると考える。$\vec{b}$と$\ket{b}$が必ずしも等価でないことに注意する。

\subsection{Outline of the article}

以下では,
\begin{enumerate}
  \item algorithm(とruntime, comparison)
  \item このalgorithmがoptimalであると示す
  \item application
\end{enumerate}

\subsection{related work}

また,related workとして制限付きでの線形操作の例や,非線形微分方程式への拡張の話題がある。

\section{Preliminary}
何書こうか迷い中


\section{Main Result}
以下では具体的なアルゴリズムについて紹介する。まず,Ref[3]で,行列$A$を$O(\log(N)s^2t)$をupper boundとする計算量で$\exp(iAt)$で計算することができる。

次に,行列$A$がHermitianかどうかで場合分けをする。

$A$がHermitianでない場合,
\begin{eqnarray}
  \tilde{A}:= \left(\begin{array}{cc}
0 & A \\
A^{\dagger} & 0
\end{array}\right)
\end{eqnarray}
と定義すれば$\tilde{A}$はHermitianであるので,方程式
$$
\tilde{A} \vec{y}=\left(\begin{array}{l}
\vec{b} \\
0
\end{array}\right)
$$
を解くことで
$$
y=\left(\begin{array}{l}
0 \\
\vec{x}
\end{array}\right)
$$
が得られる。このように,$A$がHermitianでない時,$\tilde{A}$を定義して解けるようにする一連の操作をreductionと呼ぶ。

以下では,$A$がHermitianである場合を考える。Ref[13]により,$b_{i}$ and $\sum_{i=i_{1}}^{i_{2}}\left|b_{i}\right|^{2}$ are efficiently computableならstate $\ket{b}$を任意の基底における任意の確率振幅についての重ね合わせに変換できる。ここでは$\ket{b}$を$A$の固有ベクトル(単位ベクトルを選ぶ

$$
\left|\Psi_{0}\right\rangle:=\sqrt{\frac{2}{T}} \sum_{\tau=0}^{T-1} \sin \frac{\pi\left(\tau+\frac{1}{2}\right)}{T}|\tau\rangle
$$

$\sum_{\tau=0}^{T-1}|\tau\rangle\langle\tau| \otimes e^{i A \tau t_{0} / T}$ on $\left|\Psi_{0}\right\rangle \otimes|b\rangle$, where $t_{0}=O(\kappa / \epsilon)$.

$$
\sum_{j=1}^{N} \sum_{k=0}^{T-1} \alpha_{k \mid j} \beta_{j}|k\rangle\left|u_{j}\right\rangle,
$$


$$
\sum_{j=1}^{N} \sum_{k=0}^{T-1} \alpha_{k \mid j} \beta_{j}\left|\tilde{\lambda}_{k}\right\rangle\left|u_{j}\right\rangle
$$
Adding a qubit and rotating conditioned on $\left|\tilde{\lambda}_{k}\right\rangle$ yields
$$
\sum_{j=1}^{N} \sum_{k=0}^{T-1} \alpha_{k \mid j} \beta_{j}\left|\tilde{\lambda}_{k}\right\rangle\left|u_{j}\right\rangle\left(\sqrt{1-\frac{C^{2}}{\tilde{\lambda}_{k}^{2}}}|0\rangle+\frac{C}{\tilde{\lambda}_{k}}|1\rangle\right)
$$
where $C$ is chosen to be $O(1 / \kappa) .$ We now undo the phase estimation to uncompute the $\left|\tilde{\lambda}_{k}\right\rangle .$ If the phase estimation were perfect, we would have $\alpha_{k \mid j}=1$ if $\tilde{\lambda}_{k}=\lambda_{j}$, and 0 otherwise. Assuming this for now, we obtain
$$
\sum_{j=1}^{N} \beta_{j}\left|u_{j}\right\rangle\left(\sqrt{1-\frac{C^{2}}{\lambda_{j}^{2}}}|0\rangle+\frac{C}{\lambda_{j}}|1\rangle\right)
$$
To finish the inversion, we measure the last qubit. Conditioned on seeing 1 , we have the state
$$
\sqrt{\frac{1}{\sum_{j=1}^{N} C^{2}\left|\beta_{j}\right|^{2} /\left|\lambda_{j}\right|^{2}} \sum_{j=1}^{N} \beta_{j} \frac{C}{\lambda_{j}}\left|u_{j}\right\rangle}
$$


which corresponds to $|x\rangle=\sum_{j=1}^{n} \beta_{j} \lambda_{j}^{-1}\left|u_{j}\right\rangle$ up to normalization.


\section{Details, Discussion}
何書こうか迷い中

\section{Conclusion(Summary)}
基本Introと同じ。アルゴリズムと同じ

\section{Appendix}
Quantum Fourier Transformation(QFT)についてQCQIの内容掲載

Quantum Pase Estimation(QPE)についてQCQIの内容を掲載

\section{Reference}
HHLとその参考文献





\end{document}
